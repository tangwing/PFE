
\chapter{Conclusion}
Ce projet de science de la d�cision nous permet d'avoir une id�e sur la visualisation et la fouille des donn�es.

En entr�e, nous avons 10 fichiers CSV qui contiennent environ 3,7 millions d'enregistrements d'appel.

Le traitement se concentre sur l'extraction des informations utiles et la g�n�ration des nouveaux fichiers de format sp�cifique de Tulip. Ensuit nous importons les nouveaux fichiers en Tulip pour visualiser les donn�es sous forme de graphe.

En sortie, nous obtenons des graphes qui repr�sentent les relations entre les interlocuteurs. A partir des graphes, on peut obtenir beaucoup d'informations telles que lequel r�seau de relations est le plus grand (selon la taille des sous-graphes) etc.

La plus grande acquisition que nous r�coltons c'est que nous arrivons � percevoir les difficult�s envisag�es par le traitement des donn�es volumineuses. Quand le volume de donn�es est petit, il peut y avoir plusieurs moyens pour r�aliser le travail, on peut choisir n'importe lequel, le r�sultat obtenu et le temps d'ex�cution seront � peu pr�s pareils. Mais quand le volume de donn�es est tr�s grand, beaucoup de contraintes sont apport�es, alors chaque d�cision (choix de technologie, conception de l'algorithme...) doit se faire avec tr�s attention, car avec ce volume, une petite diff�rence entre deux d�cisions va peut-�tre conduire � plusieurs jours de diff�rence sur le temps d'ex�cution.



