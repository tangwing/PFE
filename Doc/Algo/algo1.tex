%%%%%%%%%%%%%%%%%%%%%%%%%% Algo 3 (567) %%%%%%%%%%%%%%%%%%%%%%
\begin{algorithm}[H]
\caption{CréerListesTâches}
\label{algo3}
\algsetup{indent=3em}
\begin{algorithmic}[1]
\REQUIRE {
	\STATE indice: l'indice de l'intervalle sur lequel on travaille
}
\ENSURE {
	STATE ListeTachesPr: la liste des tâches préemptable.
	STATE ListeGPU1,ListeGPU2,ListeCPU1,ListeCPU2,ListePrGPU1,ListePrGPU2,ListePrCPU1,ListePrCPU2: les tâches sont réparties dans ces 8 listes selon leurs besoins en GPU, CPU, HDD, RAM et si la tâche est préemptable.
}
\LOOP
	STATE Entier temps;
	STATE temps = ListeIntervalle[indice].BorneInf;
	Pour tâche de 0 à N()-1 Faire
		\IF{(u(tâche, temps)=1) } //Cette	tâche peut être exécutée en ce moment
			\IF{(R(tâche==0))	}	//Cette tâche est non-préemptable
				\IF{(ng(tâche)>0) }
					\IF{(nh(tâche) > nr(tâche)) }
						STATE Répartir cette tâche à ListeGPU1;
					\ELSE
						STATE Répartir cette tâche à ListeGPU2;
					\ENDIF
				\ELSE	//ng(tâche)=0
					\IF{(nh(tâche) > nr(tâche)) }
						STATE Répartir cette tâche à ListeCPU1;
					\ELSE
						STATE Répartir cette tâche à ListeCPU2;
					\ENDIF
				\ENDIF
			\ELSE	//Cette tâche est préemptable
				Répartir cette tâche à ListeTâchesPr;
				\IF{(ng(tâche)>0) }
					\IF{(nh(tâche) > nr(tâche)) }
						STATE Répartir cette tâche à ListePrGPU1;
					\ELSE
						STATE Répartir cette tâche à ListePrGPU2;
					\ENDIF
				\ELSE	//ng(tâche)=0
					\IF{(nh(tâche) > nr(tâche)) }
						STATE Répartir cette tâche à ListePrCPU1;
					\ELSE
						STATE Répartir cette tâche à ListePrCPU2;
					\ENDIF
				\ENDIF
			\ENDIF
	\ENDFOR
\ENDLOOP
\end{algorithmic}
\end{algorithm} 

%%%%%%%%%%%%%%%%%%%%%%%%%%%%%%
Algo1 CaculerIntervalle()
%{}
\begin{algorithm}[H]
\caption{CaculerIntervalle}
\label{algo1}
\algsetup{indent=3em}
\begin{algorithmic}[1]
\ENSURE {
\STATE	ListeIntervalle: le tableau des intervalles trouvés
}\\
\LOOP
\STATE Entier nbIntervalle=0;
\STATE ListeIntervalle[0].BorneInf = 0;
\STATE ListeIntervalle[0].BorneSup = 0;
	\FOR{ t allant de 0 à T-2 }
		\FOR{ iTache allant de 0 à N-1 }
			%\COMMENT{Intervalle trouvé}
			\IF{ u(iTache,t)!=u(iTache,t+1) }  
				\STATE nbIntervalle = nbIntervalle+1;
				\STATE ListeIntervalle[nbIntervalle-1].BorneSup = t;
				%\COMMENT{Intervalle suivant}
				\STATE ListeIntervalle[nbIntervalle].BorneInf = t+1; 
				\STATE break;
			\ENDIF
		\ENDFOR
	\ENDFOR
	\STATE nbIntervalle=nbIntervalle+1;
	\STATE ListeIntervalle[nbIntervalle-1].BorneSup = T-1;
\ENDLOOP
\end{algorithmic}
\end{algorithm}

%%%%%%%%%%%2
%%%%%%%%%%%%%%%%%%%%%%%%%%%%%%
Algo2 CréerListeServeurTriée
\begin{algorithm}[H]
\caption{CréerListeServeurTriée}
\label{algo2}
\algsetup{indent=3em}
\begin{algorithmic}[1]
\ENSURE {
\STATE	ListeServeur: le tableau des serveurs triés par ordre croissant en fonction du coût normalisé
}\\
\LOOP
	\STATE Réel CoutTotal, SommeCaract;
	\FOR {i allant de 0 à M-1 }
		\STATE ListeServeur[i].IndiceServeur=i;
		\STATE CoutTotal=mc(i)*alphac(i) +mg(i)*alphag(i) + mr(i)*alphar(i)+ mh(i)*alphah(i);
		\STATE SommeCaract = mc(i)+mg(i)+mr(i)+mh(i);
		\STATE ListeServeur[i].CoutNorm = CoutTotal / SommeCaract ;
	\ENDFOR
	\STATE TrierParCoutNorm(ListeServeur);
\ENDLOOP
\end{algorithmic}
\end{algorithm}