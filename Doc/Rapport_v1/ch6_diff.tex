\chapter{Difficultés rencontrées}
\section{Compréhension du modèle mathématique}
Le travail au début du projet est de reprendre les éléments existants surtout le modèle mathématique qui est composé par beaucoup de formules. La création de ces formules étant déjà un gros travail, c'est surtout pas facile de comprendre l'idée derrière. Heureusement avec l'aide de mon encadrant, j'ai enfin franchi cet obstacle et assuré l'avancement du projet.

\section{Solveur Cplex}
Pendant la réalisation de ce projet, nous avons été perturbé par le solveur Cplex. En tant que logiciel commercial, nous n'avons pas de moyen pour connaître le fonctionnement intérieur du Cplex, ceci a posé certains problèmes car nous ne pouvons pas prévoir son comportement.

Par exemple Cplex a mis en place des stratégies heuristiques pendant sa résolution, malgré qu'il nous avait donné l'impression qu'il applique une méthode exacte pendant la rechercher de la solution optimale. Il existe de nombreux paramètres pour ajuster la stratégie du Cplex. En tant que logiciel assez compliqué, le réglage de ces paramètres fait déjà un gros travail indépendant donc nous ne sommes pas entrés dedans à cause de la limite du temps de ce projet.

\section{Complexité du problème}
Une autre difficulté vient de la complexité du problème que nous traitons. Pendant la deuxième partie du projet qui concerne le Preprocessing, nous avons effectué une vingtaines de tests pour analyser la performance des coupes. Mais à cause de la complexité du problème, un test a souvent besoin d'une ou deux journée pour se terminer. En plus nous ne pouvons que lancer les tests en ordre séquentielle sur la même machine pour garder la cohérence entre les tests. C'est un peu perturbant sur le bon avancement du projet. Heureusement, tout le travail important du projet a été fini avant la date de fin.


