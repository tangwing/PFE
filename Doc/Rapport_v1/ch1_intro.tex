\chapter{Introduction}
Ce projet est développé dans le cadre du projet de fin d'études (noté PFE) au département informatique, école d'ingénieurs Polytech'Tours. Le sujet, qui est de nature de recherche scientifique, consiste à résoudre un problème d'ordonnancement NP-complet rencontré par les fournisseurs d'infrastructure comme Amazon Cloud. On cherche à ordonnancer des machines virtuelles (tâches des clients) sur des machines physiques de façon optimale pour minimiser le coût d'utilisation.


Basé sur le travail existant, les travaux à réaliser sont:
\bigskip
\begin{itemize}
	\item Corriger et améliorer la méthode heuristique existante.
	\item Implémenter une autre méthode heuristique basée sur le solveur CPLEX.
	\item Effectuer des recherches sur le Preprocessing du modèle pour améliorer le fonctionnement de la méthode exacte.
	\item Effectuer des campagnes de tests.
	\item Analyser et comparer les résultats obtenus.
\end{itemize}
\bigskip


La première partie de ce rapport portera sur la présentation et la modélisation du problème, les travaux qui ont déjà été effectués seront aussi présentés dans cette partie.


La deuxième partie présentera les travaux réalisés, y compris la réalisation des méthodes de résolution, le travail sur le technique de Preprocessing ainsi que les analyses et les comparaison des résultats de tests. 


La troisième partie s'agit de la gestion du projet, y compris la planification et la gestion de versions.

On parlera à la fin sur les difficultés rencontrées et la conclusion.