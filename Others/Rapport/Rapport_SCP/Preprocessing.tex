\documentclass[twoside,fleqn]{EPURapport}
%\usepackage{listings}
\usepackage[french]{algorithm2e}
\usepackage{caption}
%\renewcommand{\lstlistlistingname}{Liste des codes}
%\renewcommand{\lstlistingname}{Code}

%\addextratables{%
%	\lstlistoflistings
%}

%\swapAuthorsAndSupervisors


\usepackage{hyperref}
\usepackage{amsmath}
\usepackage{ulem}
\usepackage[utf8]{inputenc}

\nolistoftables
\thedocument{Rapport PFE}{Résolution d'un problème off-line de consolidation de serveurs}{}

\grade{Département Informatique\\ 5\ieme{} année\\ 2013 - 2014}
\authors{%
	\category{Étudiants}{%
		\name{Lei SHANG} \mail{lei.shang@etu.univ-tours.fr}
	}
	\details{DI5 2013 - 2014}
}

\supervisors{%
	\category{Encadrants}{%
		\name{Vincent T'Kindt} \mail{vincent.tkindt@univ-tours.fr}
	}
	\details{Université François-Rabelais, Tours}
}

\abstracts{}{}


\renewcommand{\theequation}{\Alph{equation}}	
\begin{document}

\chapter{Cuts}%%%%%%%%%%%%%%%%%%%%%%%%%%%%%%%%%%%%%%%%%%%%%%%%%%%%%%%%%%%%%%%%%%%%%%%%%%%%%%%%%%%
\section{Cuts classiques}
Les Cuts classiques sont des coupes dont la génération ne dépends pas la particularité du problème. Nous présentons dans cette section qu'un seul type de coupes: 1-Cuts.
\subsection{1-Cuts}
1-cuts par Osorio et al.(2002) sont des coupes générées à partir des contraintes de types $d^Tx \leq b$ avec $d_1  \geq d_2 \geq \dots \geq d_n > 0$. Ce sont donc des contraintes redondantes qui peut pourtant plus efficaces. Par exemple pour la contrainte $x_1+2x_2+2x_3\leq 3 \ $dont les variables sont binaires, on peut en déduire un 1-cuts $x_2+x_3 < 1$, car si $x_2 = x_3=1$ la contrainte originale sera violée.

Il existe déjà l'algorithme\footnote{V'Tkindt et al, 2007} pour générer automatiquement les 1-cuts, alors nous avons réalisé cet algorithme à partir du pseudo-code. Nous avons ensuite appliqué cet algorithme sur les contraintes de ressources dans le Preprocessing. Le résultat de test montre que ces coupes ont bien un effet possitif pour fixer plus de variables surtout pour les permiers 3 scénarios. Cependant, nous avons aussi aperçu que le nombre de coupes générées est considérable pendant cette démarche, ce qui peut potentiellement avoir un effet négatif sur le temps de résolution, parce que quand nous avons de nombreux contraintes ajoutés, Cplex va alors mettre plus de temps pour traiter ces contraintes.

Pour résoudre ce problème, nous nous posons la question: combien de 1-cuts devons nous générer et quelles sont les contraintes prioritaires. Empiriquement, nous décitons de trier les contraintes par ordre croissante de la partie droite d'équation (noté RHS) car quand le RHS est plus petit, ça génère des coupes plus contraignant. Pour la question sur le nombre de coupes à générer, nous avons fait un test sur des instances choisies avec un nombre de coupes différent pour voir c'est combien le seuil pour chaque scénario. Ensuite avec le résultat des scénarios 4, 5 et 6 comme échantillons, nous avons trouvé une fonction qui peut donner un seuil selon le nombre de tâche et le nombre de machine du problème. La recherche de cette fonction est basé sur le procédure de "Multiple Linear Regression". Un outil en ligne\footnote{\url{http://www.xuru.org/rt/MLR.asp}} a été utilisé pour cette recherche.

Les différents tests effectués sont décrits dans la section \ref{Test et Analyse}

\section{Cuts problème-dépendent}
\subsection{Cuts sur les contraintes des ressources}
\subsubsection{Ressources CPU/GPU}
Si la tâche $i$ ne peut pas être affectée au serveur $j$ à cause de la capacité résiduelle de CPU/GPU du serveur, alors pour toute les tâches qui demandent plus de CPU/GPU que la tâche $i$, cette affectation ne peut pas effectuée non plus.


Cette contrainte peut être exprimée de façon suivante:
\bigskip

%CPU
$Si \  n^c_{i\prime}\geq n^c_{i}\ alors\;$
\begin{align} 
&x_{i,j,t}+x_{i\prime,j,t}\leq (m^c_j-\sum^N_{k=1; k\neq{i},i\prime;u_{k,t}=q_{k,j}=1}{n^c_kx_{k,j,t}})/n^c_i
&&\forall t=1,\ldots,T, tq\ u_{i,t}=u_{i\prime,t}=1; \nonumber \\
 & &&\forall j=1, \ldots, M, tq\ q_{i,j}=q_{i\prime,j}=1
\end{align} 
 
%GPU
$Si\ n^g_{i\prime}\geq n^g_{i}\ alors\;$
\begin{align} 
&x_{i,j,t}+x_{i\prime,j,t}\leq (m^g_j-\sum^N_{k=1; k\neq{i},i\prime;u_{k,t}=q_{k,j}=1}{n^g_kx_{k,j,t}})/n^g_i 
&&\forall t=1,\ldots,T, tq\ u_{i,t}=u_{i\prime,t}=1; \nonumber \\
 & &&\forall j=1, \ldots, M, tq\ q_{i,j}=q_{i\prime,j}=1
\end{align} 


À noter que nous n'avons pas besoin de considérer ici la contraite de préaffectation.

\subsubsection{Ressources HDD/RAM}
Les cuts sur les ressources HDD/RAM ont le même principe sauf que ces ressources puissent aussi être occupées par l'opération de la migration. Nous pouvons appliquer les mêmes cuts comme pour CPU/GPU mais la prise en compte de la migration peut rendre le cut plus strict.
\bigskip

%HDD
$Si\ n^h_{i\prime}\geq n^h_{i}\ alors\;$
\begin{align}
x_{i,j,t}+x_{i\prime,j,t} &\leq (m^h_j-\sum^N_{k=1;k\neq{i},i\prime;u_{k,t}= q_{k,j}=1}{n^h_kx_{k,j,t}} \nonumber \\
 & - \sum^N_{k=1; k\neq{i},i\prime}{\sum^M_{l=1;l\neq{j}}{n^h_ky^{l,j}_{k,k,t}} }\;)/n^h_i        &&\forall t=1,\ldots,T, tq\ u_{i,t}=u_{i\prime,t}=1;  \nonumber \\
 & &&\forall j=1, \ldots, M, tq\ q_{i,j}=q_{i\prime,j}=1
\end{align}

%RAM
$Si\ n^r_{i\prime}\geq n^r_{i}\ alors\;$
\begin{align}
x_{i,j,t}+x_{i\prime,j,t} &\leq (m^r_j-\sum^N_{k=1;k\neq{i},i\prime;u_{k,t}= q_{k,j}=1}{n^r_kx_{k,j,t}} \nonumber \\
 & - \sum^N_{k=1; k\neq{i},i\prime}{\sum^M_{l=1;l\neq{j}}{n^r_ky^{l,j}_{k,k,t}} }\;)/n^r_i        &&\forall t=1,\ldots,T, tq\ u_{i,t}=u_{i\prime,t}=1;  \nonumber \\
 & &&\forall j=1, \ldots, M, tq\ q_{i,j}=q_{i\prime,j}=1
\end{align}


\subsection{Machines équivalentes*}
Deux serveurs $j$ et $j'$ peut être totalement identique dans une instance de problème si:
\begin{enumerate}
\item Les caractéristiques (CPU/GPU/HDD/RAM) de $j$ et $j'$ sont identiques
\item $q_{i,j} = q'_{i,j}$ pour $\forall i$
\item \sout{$j$ et $j'$ ont les même voisins dans le réseau}
\item \sout{Pour chaque voisin $v$ de $j$ (et $j'$), la bande passante entre $v$ et $j$ est la même que celle entre $v$ et $j'$}
\end{enumerate}
\bigskip
\sout{Plus largement, le $j$ et $j'$ ci-dessus peuvent être considérés comme un sous-réseau mais pas simplement un seul serveur. Comme ça, étant donné une solution, on peut très bien construir une autre en échangeant les affectation en $j$ et $j'$.}

[Update]Après étude du réseau, il parâit que toutes les machines sont connectées et la bande passante est unique donc on n'a besoin que de considérer les deux premières conditions. [Update]Mais quand même, si l'affectation ne se fait pas au premier instant, c'est possible que l'environnement réseau des machines équivalentes devient différent.

Si $j$ et $j'$ sont deux machines identiques alors pour le premier instant de temps où il y a des tâches qui s'exécutent sur $j$ ou $j'$, alors on peut forcer que le serveur $j$ est plus utilisé que $j'$ pour éliminer les solutions redondantes:
\begin{align}
\sum_{t_1=1}^{t-1}\sum_{i_1=1}^{N}x_{i_1,j,t_1}+ CPUUsed(j,t)-CPUUsed(j',t) \geq 0
 && \forall t=1,\ldots,T;   \nonumber \\
 && \forall j,j'=1\ldots M, j<j', \nonumber \\
 && équivalent(j,j');
\end{align}
[Update]C'est faux! On ne doit pas utiliser la différence des CPUUsed dans le formule car cette différence n'est pas binaire, donc même si $\sum_{t_1=1}^{t-1}\sum_{i_1=1}^{N}x_{i_1,j,t_1} \geq 0$, cette contrainte est quand même fonctionnelle.

\subsection{Dominance des tâches* (à voir)}
Cette fois on va prendre en compte toutes les ressources requises par la tâche. On modélise 2 tâches $i$ et $i\prime$ pour que la tâche $i\prime$ a besoin de plus de ressource que $i$ pour tout type de ressources y compris la partie du réseau.

La formulation courante de ce cut n'est peut-être pas valide, parce que même si nous avons $x_{i,j,t}=0$ ça ne veut pas dire qu'il n'y a pas de ressources pour la tâche $i$. Ça peut simplement parce que $x_{i,j,t}=1$ ne conduit pas à la solution optimale.

Alors si nous voulons créer un Cut sur les ressources du réseau, je pense c'est quand même mieux de reprendre le même principe: si on est sûr que les ressources réseau sont insuffisantes pour la tâche $i$ alors c'est pareil pour $i\prime$. Cependant cette modélisation (état de l'insuffisance de ressources) ne me semble pas évidente.


\chapter{Test des 3 méthodes de résolution sans Preprocessing}%%%%%%%%%%%%%%%%%%%%%%%%%%%%%%%%%%%%
Pour tester le fonctionnement des différentes méthodes de résolution dans ce projet, plusieurs tests ont été effectués sur un ensemble d'instance du problème. Cet ensemble contient 6 scénarios dont chacun est composé par 20 instances du problème. Toutes ces données de test sont générées par le programme Testeur.

%\section{Test des 3 méthodes de résolution}
\section{Méthode exacte - solveur Cplex}
Dans un premier temps, le test du solveur Cplex est effectué car les solutions trouvées par le solveur peuvent nous servir à évaluer la performance des autres méthodes de résolution.

 
Le tableau \ref{tab_cplex} représente les statistiques effectuées sur le résultat de test du solveur Cplex. Les significations des colonnes sont:
\begin{enumerate}
	\item Sc(N/M): numéro de scénario et le nombre de VM et de serveur physique.
	\item \#Infeas: le nombre des instances qui sont prouvées comme ''Infaisable'' par le solveur.
	\item \#SolvedOpt: le nombre des instances qui sont résolues avec la solution optimale trouvée.
	\item \#Mem: le nombre des instances pour lesquelles le solveur n'a pas pu trouver la solution optimale à cause de la limite de l'espace mémoire.
	\item \#Tim: le nombre des instances pour lesquelles le solveur n'a pas pu trouver la solution optimale à cause de la limite du temps.
	\item $T_{min}$, $T_{avg}$, $T_{max}$: le temps (minimum, moyenne et maximum) de résolution en seconde.
\end{enumerate}
\bigskip


\begin{table}[h]
    \centering
    \begin{tabular}{|c|c|c|c|c|c|c|c|}
    	\hline
    	Sc(N/M)	& \#Infeas & \#SolvedOpt	& \#Mem & \#Tim & $T_{min}$ & $T_{avg}$	& $T_{max}$ \\ \hline
		Sc1(8/2) & 2 & 18 & 0 & 0 &  0.02 &  0.07 &  0.15 \\ \hline
		Sc2(11/3) & 9 & 11 & 0 & 0 &  0.08 &  0.34 &  1.08 \\ \hline
		Sc3(15/4) & 1 & 19 & 0 & 0 &  0.59 &  4.33 &  54.57 \\ \hline
		Sc4(18/5) & 0 & 19 & 0 & 1 &  1.88 &  239.53 &  1800.37 \\ \hline
		Sc5(21/5) & 3 & 6 & 1 & 10 &  1.68 &  1196.57 &  1800.74 \\ \hline
		Sc6(24/6) & 2 & 5 & 0 & 13 &  2.06 &  1316.75 &  1820.14 \\	\hline
    \end{tabular}
    \label{tab_cplex}
    \caption{Résultat de test du solveur Cplex}
\end{table}
\bigskip

A partir du scénario 4, on commence à avoir des grosses instances pour lesquelles le solveur n'a pas pu trouver la solution optimale ($T_{max}=1800$) à cause de la limite de temps.

\section{Méthode heuristique de liste}
Le tableau \ref{tab_h1} montre le résultat de test de la méthode heuristique de liste (on l'appelle H1 pour raison de simplicité). Dans ce tableau, les colonnes $D_{min}$, $D_{avg}$ et $D_{max}$ signifient la déviation entre la solution trouvée par H1 et la solution trouvée par le solveur Cplex. A noter que pour les 3 premiers scénarios le solveur a trouvé la solution optimale pour toutes les instances donc cette déviation peut montrer la qualité de notre méthode H1 par rapport à la solution optimale. A partir du scénario 4 on commence à avoir des instances pour lesquelles le solveur a trouvé une solution faisable mais pas optimale à cause de la limite du temps ou de l'espace mémoire, alors dans ce cas la déviation est aussi affectée.


\begin{table}[h]
    \centering
    \begin{tabular}{|c|c|c|c|c|c|c|c|c|}
    	\hline
    	Sc(N/M)	& \#Infeas & \#Solved	& $T_{min}$ & $T_{avg}$	& $T_{max}$ & $D_{min}$ & $D_{avg}$	& $D_{max}$ \\ \hline
		Sc1(8/2)  & 2 & 18 & 0.00 & 0.00 & 0.00 &0\% &19\% &67\% \\ \hline
Sc2(11/3) & 9 & 11 & 0.00 & 0.00 & 0.00 &0\% &17\% &46\% \\ \hline
Sc3(15/4) & 7 & 13 & 0.00 & 0.00 & 0.00 &5\% &24\% &58\% \\ \hline
Sc4(18/5) & 11 & 9 & 0.00 & 0.00 & 0.00 &10\%& 27\%& 37\% \\ \hline
Sc5(21/5) & 16 & 4 & 0.00 & 0.00 & 0.01 &30\%& 36\%& 43\% \\ \hline
Sc6(24/6) & 20 & 0 & 0.00 & 0.01 & 0.02 & * & * & * \\ \hline
    \end{tabular}
    \label{tab_h1}
    \caption{Résultat de test de la méthode heuristique de liste (H1)}
\end{table}
\bigskip

On trouve souvent 0.00 seconde dans les colonnes de temps, ce qui signifie que le temps de résolution de H1 est très court.


Par rapport à la déviation, la performance de H1 n'est pas très stable: pour certaines instances du problème, H1 a trouvé la solution optimale ($D_{min}=0\%$) mais on a aussi dans le Sc1 $D_{max}=67\%$ qui n'est pas très optimiste. Pour le scénario 6, H1 n'a résolu aucune instance donc la déviation n'est pas calculée.


\section{Méthode heuristique de Cplex}
Le tableau \ref{tab_h2} montre le résulat de test sur la méthode heuristique basée sur le solveur Cplex (H2).


\begin{table}[h]
    \centering
    \begin{tabular}{|c|c|c|c|c|c|c|c|c|}
    	\hline
    	Sc(N/M)	& \#Infeas & \#Solved	& $T_{min}$ & $T_{avg}$	& $T_{max}$ & $D_{min}$ & $D_{avg}$	& $D_{max}$ \\ \hline
		Sc1(8/2)  &2 & 18 &  0.02 &  0.07   &0.15     &0\%  &0\%  &1\% \\ \hline
Sc2(11/3) &9 & 11 &  0.08 &  0.29   &0.97     &0\%  &0\%  &1\% \\ \hline
Sc3(15/4) &1 & 19 &  0.64 &  1.37   &5.21     &0\%  &0\%  &2\% \\ \hline
Sc4(18/5) &0 & 20 &  1.53 &  80.33  & 400.32  &0\%  &0\%  &2\% \\ \hline
Sc5(21/5) &3 & 17 &  1.68 &  240.31 &  400.41 & 0\% & 2\% & 4\% \\ \hline
Sc6(24/6) &2 & 18 &  2.08 &  286.79 &  410.43 & -1\%&  1\%&  7\% \\ \hline
    \end{tabular}
    \label{tab_h2}
    \caption{Résultat de test de la méthode heuristique de Cplex (H2)}
\end{table}
\bigskip


Puisque la méthode H2 est basée sur le solveur Cplex, pour les petites instances de problème elle a trouvé des solutions qui sont très proches des solutions optimales. Pour les grandes instances (à partir du scénario 4), les solutions qu'elle a trouvées sont aussi très intéressantes avec la déviation maximale égale à 7\% qui reste acceptable. Le temps maximum de résolution est vers 400 secondes qui provient de l'implémentation de H2.


\chapter{Test du Preprocessing}
Un autre ensemble de tests a été réalisé pour évaluer la performance de l'approche Preprocessing avec des coupes différentes.

On rappelle ici l'idée du Preprocessing est de fixer autant que possible de variables dans le modèle pour réduire la taille du problème. Pour le faire, il faut avoir une borne supérieure (UB) et une borne inférieure (LB) qui sont assez proches de la solution optimale.
\section{Preprocessing sans coupes supplémentaires}
Nous avons lancé d'abord le Preprocessing sur toutes les variables booléennes sans ajoutant des coupes supplémentaires.
Les colonnes s'interprêtent comme le suivant:
\begin{enumerate}
	\item $DevLB_{min}$, $DevLB_{avg}$, $DevLB_{max}$: la déviation (minimum, moyenne et maximum) entre la solution optimale et LB.
	\item $DevUB_{min}$, $DevUB_{avg}$, $DevUB_{max}$: la déviation (minimum, moyenne et maximum) entre UB et la solution optimale.
	\item $\%Fixed_{min}$, $\%Fixed_{avg}$, $\%Fixed_{max}$: la proportion du nombre de variables fixées par rapport au nombre de toutes les variables ayant passé le preprocessing.
\end{enumerate}

%\subsubsection{Preprocessing sur toutes les variables booléennes}
%Le premier test concerne le Preprocessing sur toutes les variables booléennes.


Dans le tableau \ref{tab_pre} on peut trouver des statistiques sur la qualité des LB et des UB ainsi que la proportion de variables fixées pendant le Preprocessing.
\bigskip
\begin{table}[h]
    \centering
    \begin{tabular}{|c|c|c|c|c|c|c|c|c|c|}
    	\hline
    	Sc(N/M)	& $LB_{min}$ & $LB_{avg}$ & $LB_{max}$ & $UB_{min}$ & $UB_{avg}$ & $UB_{max}$ & $\%Fix_{min}$ & $\%Fix_{avg}$ & $\%Fix_{max}$\\ \hline
Sc1(8/2) & 0.00\%  &4.83\%  &17.07\%  &0.00\% & 0.04\%  &0.65\%  &0.00\%  &9.25\% &51.83\% \\ \hline
Sc2(11/3)&  0.04\% & 7.21\% & 17.13\% & 0.00\%&  0.01\% & 0.15\% & 0.00\% &13.90\% & 66.67\%\\ \hline
Sc3(15/4)&  1.10\% & 4.01\% & 10.12\% & 0.00\%&  0.36\% & 1.80\% & 0.00\% &0.88\% & 10.96\%\\ \hline
Sc4(18/5)&  0.05\% & 5.28\% & 12.56\% & 0.00\%&  0.53\% & 1.81\% & 0.00\% &2.86\% & 55.64\%\\ \hline
Sc5(21/5)&  1.82\% & 5.03\% & 8.94\%  &0.56\% & 1.94\%  &7.62\%  &0.00\%  &0.00\% &0.00\%\\ \hline
Sc6(24/6)&  3.94\% & 5.91\% & 8.72\%  &0.30\% & 1.75\%  &5.40\%  &0.00\%  &0.09\% &0.56\%\\ \hline
    \end{tabular}
    \label{tab_pre}
    \caption{Résultat de test du Preprocessing sur toute les variables booléannes}
\end{table}
\bigskip
Nous constatons que le Preprocessing naïve ne fonctionne pas bien. Il ne fixe pas beaucoup de variables donc il n'aide pas la résolution de Cplex. Souvent, ce problème survient quand l'UB ou LB n'est pas assez bonne. En comparant la qualité de LB et de UB, on peut trouver que relativement les UB sont assez proches que la solution optimale, en revanche les LB ne sont pas très bonnes. 

Pour améliorer les LB, nous avons cherché d'ajouter des contraintes supplémentaires (Cuts) au modèle LP du problème, pour enfin améliorer le fonctionnement du Preprocessing.

\begin{comment}
\subsubsection{Preprocessing sur les variables booléennes de décision $X^t_{i,j}$}
Le tableau \ref{tab_pre_x} est le résultat d'un autre test sur Preprocessing. Dans ce test, au lieu de passer toutes les variables au Preprocessing, on passe seulement les variables de décision $X^t_{i,j}$, qui représente l'ordonnancement trouvé, au Preprocessing car on pense que ces variables sont les plus influentes. Par rapport au premier test, ce test ne change que les 3 dernières colonnes sur la proportion des variables fixées, le résultat sur la qualité de LB et UB reste le même.


\begin{table}[h]
    \centering
    \begin{tabular}{|c|c|c|c|}
    	\hline
    	Sc(N/M)	&  $\%Fix_{min}$ & $\%Fix_{avg}$ & $\%Fix_{max}$\\ \hline
    	Sc1(8/2) &0.00\% & 13.84\% & 100.00\%\\ \hline
Sc2(11/3)&0.00\% & 10.15\% & 42.42\%\\ \hline
Sc3(15/4)&0.00\% & 0.56\%  &7.36\%\\ \hline
Sc4(18/5)&0.00\% & 3.07\%  &59.07\%\\ \hline
Sc5(21/5)&0.00\% & 0.00\%  &0.00\%\\ \hline
Sc6(24/6)&0.00\% & 3.45\%  &59.07\%\\ \hline
    \end{tabular}
    \label{tab_pre_x}
    \caption{Résultat de test du Preprocessing sur les variables de décision $X^t_{i,j}$}
\end{table}
\bigskip

\subsection{Conclusion sur le Preprocessing}
En comparant le résultat détaillé des deux tests effectués, on a constaté que la plupart des variables fixées sont les variables de décision $X^t_{i,j}$. Cependant, pour certaines instances, si on passe seulement les $X^t_{i,j}$ au Preprocessing aucune variable peut être fixée mais si on passe toutes variables booléennes au Preprocessing il y aura des variables fixées. Ça veut dire il y a quand même des variables booléennes qui ne sont pas $X^t_{i,j}$ mais qui ont des effets sur le Preprocessing. En conclusion on pense que c'est mieux de faire le Preprocessin pour toutes les variables booléennes.

\end{comment}

\section{Preprocessing avec coupe 1}
Nous avons d'abord relancé le test de Preprocessing avec la coupe 1 qui est problème-dépendant concernant les contraintes de ressources. Malheureusement, l'ajout de cette coupe n'a apporté aucun changement au résulat de test (identique que \ref{tab_pre}).

\section{Preprocessing avec coupe 2}
Cette partie contient le résultat de test de Preprocessing avec la coupe 2 (aussi notée 1-Cuts).
\clearpage
\subsection{Premier test de la coupe 2}
\begin{table}[h]
    \centering
    \begin{tabular}{|c|c|c|c|c|c|c|c|c|c|}
    	\hline
Sc(N/M)	& $LB_{min}$ & $LB_{avg}$ & $LB_{max}$ & $UB_{min}$ & $UB_{avg}$ & $UB_{max}$ & $\%Fix_{min}$ & $\%Fix_{avg}$ & $\%Fix_{max}$\\ \hline
Sc1(8/2) & 0.00 \% &	2.87\%  &	17.07	\%  &0.00\% & 0.04\%  &0.65\%  &0.00\%  &43.91	\% &100.00\% \\ \hline
Sc2(11/3)& 0.04 \% & 	4.54\% & 	11.97	\% & 0.00\%&  0.01\% & 0.15\% & 0.00\%  &15.80	\% &66.67\%\\ \hline
Sc3(15/4)& 0.44 \% & 	3.47\% & 	9.40	\% & 0.00\%&  0.36\% & 1.80\% & 0.00\%  &2.22	\% &10.96\%\\ \hline
Sc4(18/5)& 0.05 \% & 	4.94\% & 	11.60	\% & 0.00\%&  0.53\% & 1.81\% & 0.00\%  &2.91	\% &55.64\%\\ \hline
Sc5(21/5)& 1.82 \% & 	4.72\% & 	8.82	\%  &0.56\% & 1.94\%  &7.62\%  &0.00\%  &0.00	\% &0.00\%\\ \hline
Sc6(24/6)& 3.64 \% & 	5.65\% & 	8.45	\%  &0.30\% & 1.75\%  &5.40\%  &0.00\%  &0.09	\% &0.56\%\\ \hline 
    \end{tabular}
    \label{tab_pre_2}
    \caption{Résultat de test du Preprocessing avec la coupe 2}
\end{table}
\bigskip

%\subsubsection{Fixage de variable}
Dans le résultat, les trois colonnes de UB restent les mêmes car le Preprocessing n'affecte que sur les LB. Nous pouvons trouver que avec la coupe 2, le Preprocessing a pu fixé plus de variables qu'avant, surtout pour les premiers 3 scénarios. En conséquence, les LB ont été aussi améliorées (la déviation entre la solution optimale et la LB devient plus petite).

En résumé, la coupe 2 est utile pour renforcer le Preprocessing mais le résultat n'est pas encore assez bon pour bien accélérer la résolution après. A partir de scénario 4, le Preprocessing n'aide quasiment plus.


\subsection{Deuxième test de la coupe 2 (avec seuil)}
Dans la phase de fixage, le Preprocessing peut fixer des variables en ajoutant des contraintes, mais quand le nombre de contraintes devient plus en plus grand, il va au contraire ralentir la résolution de Cplex dans la phase suivante. Dans notre cas, puisque la coupe 2 va générer beaucoup de contraintes, nous essayons donc de trouver un seuil supérieur pour le nombre de contraintes généré. Mise en place de ce paramètre doit diminuler autant que possible le temps utilisé pour la résolution MIP, tout en assurant que le nombre de variables fixées ne soit pas affecté.

Pour trouver la relation entre le nombre de contraites générées à partir de la coupe 2 et le temps dépensé pendant la résolution Cplex, nous avons conçu deux test d'analyse.
\subsubsection{Premier test de l'analyse du nombre de coupes à générer}
Les principles du premier test sont:
\begin{enumerate}
	\item Pour chacun des scénarios 4, 5 et 6, nous choisissons 5 instances de problème qui peuvent être résolues à l'optimalité dans un temps modéré (pas trop court pour être observable ni trop long pour ne pas dépasser la limite de temps).
	\item Pour chaque instance, nous faisons un série de tests avec un seuil de nombre de coupes différents. Ce seuil commence à 200 puis s'incrémente d'un pas de 200 jusqu'à le nombre total des coupes qu'on peut générer.
	\item Après le test, pour chaque scénario, nous cherchons le seuil qui donne un temps moyen d'exécution minimum pour les 5 instances choisies.
	\item Pour que les coupes générées soient les plus contraignantes, nous avons trier avant la génération de coupe 2, les contraintes en ordre croissante de RHS (cf \ref{}).
\end{enumerate}
	
Le résultat du test nous a donné des échantillons pour étudier la corrélation entre le meilleur seuil et les caractéristiques de scénarios.

\begin{table}[h]
    \centering
    \begin{tabular}{|c|c|c|c|}
    	\hline
Sc& 	N	& M	& BestSeuil\\ \hline
4 & 	18	& 5	& 600      \\ \hline
5 & 	21	& 5	& 400      \\ \hline
6 & 	24	& 6	& 1600     \\ \hline
    \end{tabular}
    \label{tab_pre_2_seuil}
    \caption{Résultat du premier test d'analyse du seuil de la coupe 2}
\end{table}
\bigskip

Ce résultat montre que quand N (le nombre de tâches) augmente, BestSeuil diminue et quand M (le nombre de machines) augmente, BestSeuil augmente. Empiriquement, nous avons fait une "Multiple Linear Regression"\footnote{\url{http://www.xuru.org/mlr}} sur ces données de test pour trouver une fonction qui peut donner le meilleur seuil pour un scénario. Le fonction trouvée est:
\begin{align}
BestSeuil=-66.67N+1400M-5200 
\end{align}

Elle s'applique pour les scénarios après 4, car pour les 3 premiers le nombre de coupes n'est pas grand, le temps de résolution n'est aussi négligeable.

Néanmoins, la corrélation entre le temps écoulé et le nombre de contraintes n'est pas très fort dans le cas de Cplex. Ça concerne le mécanisme au sein de Cplex que nous ne pouvons pas savoir donc cet approche est avant tout empirique.

% LHS/RHS
\subsubsection{Deuxième test de l'analyse du nombre de coupes à générer}
Une deuxième analyse a été faite pour chercher le meilleur seuil de la coupe 2. Cet analyse est presque pareil que la première sauf que cette fois au lieu de trier les contraintes en ordre croissante de RHS, nous les trions en ordre décroissante de LHS/RHS, donc LHS signifie le somme de tous les coefficients à gauche de l'équation.

\begin{table}[h]
    \centering
    \begin{tabular}{|c|c|c|c|}
    	\hline
Sc& 	N	& M	& BestSeuil\\ \hline
4 & 	18	& 5	& 1800      \\ \hline
5 & 	21	& 5	& 1200      \\ \hline
6 & 	24	& 6	& 2600     \\ \hline
    \end{tabular}
    \label{tab_pre_2_seuil2}
    \caption{Résultat du deuxième test d'analyse du seuil de la coupe 2}
\end{table}
\bigskip

La fonction trouvée cette fois est:
\begin{align}
BestSeuil=-200N + 2000M-4600 %&&(seuil\_func 2)
\end{align}

% Test de cut2 avec seuil.
\subsubsection{Test de la coupe 2 avec seuil }
Après mettre en place la fonction de seuil, nous avons relancé encore 2 fois le test de la coupe 2 avec la fonction de seuil différente.

D'abord sur le fixage des variables, avec la mise en oeuvre du seuil, le nombre de variables fixées reste le même pour tous les 6 scénarios, sauf 3 instances qui sont affectées par la première fonction de seuil(A):
\clearpage
\begin{table}[h]
    \centering
    \begin{tabular}{|c|c|c|c|}
    	\hline
Sc-id& 	NumFixNoSeuil	& NumFixSeuil1	& NumFixSeuil2\\ \hline
3-11 & 	485	& 484	& 485      \\ \hline
3-19 & 	677	& 666	& 677      \\ \hline
4-8 & 	101	& 0	& 101     \\ \hline
    \end{tabular}
    \label{tab_cut2_seuil_fix_cmp}
    \caption{Résultat du deuxième test d'analyse du seuil de la coupe 2}
\end{table}
\bigskip

Ensuite sur le temps total de la résolution, voici la déviation entre les 2 cas de seuil et le cas sans seuil:
\begin{table}[h]
    \centering
    \begin{tabular}{|l|c|c|c|c|c|c|}
    	\hline
  &\multicolumn{3}{c}{Func\_Seuil1}	&\multicolumn{3}{|c|}{Func\_Seuil2}\\ \hline
Sc  & 	DevMin	& DevMax	& DevAvg& 	DevMin	& DevMax	& DevAvg \\ \hline
4 & 	-42.53\%	& 39.40\%	& -6.54\% & -42.75\% & 23.02 \%& -4.16  \%    \\ \hline
5 & 	-56.47\%	& 48.07\%	& -6.96\% & -34.76\% & 179.05\% & 10.66 \%     \\ \hline
6  & 	-50.45\%	& 184.5\%	& 4.8  \%& -66.12 \%& 141.06 \%& 2.89   \%  \\ \hline
    \end{tabular}
    \label{tab_cut2_seuil_tim_cmp}
    \caption{Déviation de temps avec différentes fonctions de seuil}
\end{table}
\bigskip

Oui...Il n'y a pas vraiment une amélioration... Nous avons constaté que dans chaque scénario il y a une ou deux instances très particulières qui perturbent l'analyse. Alors pour chaque scénario, si nous enlevons l'instance qui a une déviation la plus petite et l'autre qui a une déviation la plus grande, voisi le nouveau résultat:
\begin{table}[h]
    \centering
    \begin{tabular}{|l|c|c|c|c|c|c|}
    	\hline
  &\multicolumn{3}{c}{Func\_Seuil1}	&\multicolumn{3}{|c|}{Func\_Seuil2}\\ \hline
Sc  & 	DevMin	& DevMax	& DevAvg& 	DevMin	& DevMax	& DevAvg \\ \hline
4 & -37,86 \% &9,47  \% & -7,09 \% &-28,43\% & 21,52\% & -3,53\% \\ \hline
5 & -33,35 \% &22,21 \% & -7,75 \% &-32,93\% & 23,78\% & -6,90\% \\ \hline
6 &-49,77  \% &51,32 \% & -7,64 \% &-29,86\% & 29,26\% & -4,03\% \\ \hline
    \end{tabular}
    \label{tab_cut2_seuil_tim_cmp_modife}
    \caption{Déviation de temps avec différentes fonctions de seuil}
\end{table}
\bigskip
%\clearpage

\subsection{Preprocessing avec coupe 1\&2}
Le résultat du test de Preprocessing avec à la fois la coupe 1 et la coupe 2 reste pareil que le Preprocessing avec la coupe 2 seule. Ce fait a confirmé que la coupe 1 n'est pas valide.

\subsection{Preprocessing avec coupe 3}
La coupe 3 génère très peu de contrainte, mais, ça ne marche pas non plus...hhhhh

\end{document}


